\documentclass[onecolumn, draftclsnofoot,10pt, compsoc]{IEEEtran}
\usepackage{graphicx}
\usepackage{url}
\usepackage{setspace}
\usepackage{cite}

\usepackage{geometry}
\geometry{textheight=9.5in, textwidth=7in}

% 1. Fill in these details
\def \CapstoneTeamName{		Remote Seed}
\def \CapstoneTeamNumber{		12}
\def \GroupMemberOne{			Quanah Green}
\def \GroupMemberTwo{			Alex Ruef}
\def \GroupMemberThree{			Ethan Takla}
\def \CapstoneProjectName{		Remote Seed Identification}
\def \CapstoneSponsorCompany{	Crop and Soil Science Department, OSU}
\def \CapstoneSponsorPerson{		Dan Curry}

% 2. Uncomment the appropriate line below so that the document type works
\def \DocType{		%Problem Statement
				%Requirements Document
				%Technology Review
				%Design Document
				Progress Report
				}
			
\newcommand{\NameSigPair}[1]{\par
\makebox[2.75in][r]{#1} \hfil 	\makebox[3.25in]{\makebox[2.25in]{\hrulefill} \hfill		\makebox[.75in]{\hrulefill}}
\par\vspace{-12pt} \textit{\tiny\noindent
\makebox[2.75in]{} \hfil		\makebox[3.25in]{\makebox[2.25in][r]{Signature} \hfill	\makebox[.75in][r]{Date}}}}
% 3. If the document is not to be signed, uncomment the RENEWcommand below
\renewcommand{\NameSigPair}[1]{#1}

%%%%%%%%%%%%%%%%%%%%%%%%%%%%%%%%%%%%%%%
\begin{document}
\begin{titlepage}
    \pagenumbering{gobble}
    \begin{singlespace}
    	%\includegraphics[height=4cm]{coe_v_spot1}
        \hfill 
        % 4. If you have a logo, use this includegraphics command to put it on the coversheet.
        %\includegraphics[height=4cm]{CompanyLogo}   
        \par\vspace{.2in}
        \centering
        \scshape{
            \huge CS Capstone \DocType \par
            {\large\today}\par
            \vspace{.5in}
            \textbf{\Huge\CapstoneProjectName}\par
            \vfill
            {\large Prepared for}\par
            \Huge \CapstoneSponsorCompany\par
            \vspace{5pt}
            {\Large\NameSigPair{\CapstoneSponsorPerson}\par}
            {\large Prepared by }\par
            Group\CapstoneTeamNumber\par
            % 5. comment out the line below this one if you do not wish to name your team
            \CapstoneTeamName\par 
            \vspace{5pt}
            {\Large
                \NameSigPair{\GroupMemberOne}\par
                \NameSigPair{\GroupMemberTwo}\par
                \NameSigPair{\GroupMemberThree}\par
            }
            \vspace{20pt}
        }
        \begin{abstract}
        		

        \end{abstract}     
    \end{singlespace}
\end{titlepage}
\newpage
\pagenumbering{arabic}
\tableofcontents
% 7. uncomment this (if applicable). Consider adding a page break.
%\listoffigures
%\listoftables
\clearpage
\section{Mobile Application}
	\subsection{Where We Are}
		At its core the app will be used to send images to the server, because of this I decided to tackle camera manipulation first.
		For our purposes there are two options for including the camera, the phones camera application can be called or the camera can be integrated directly into the app.
		Most of our activities do not make full use of the screen real estate available so we decided to integrate the camera into our app.
		Now that we could directly access the camera in the app we made the camera preview the central activity, much like Snapchat.
		Once the camera preview was setup, getting an image to send was an easy task and it is conveniently passed in as a byte array.
		We also allow users to select images from their phones gallery; this is much easier because Android can call on the phones native gallery and pass the image back to our code.

		Now that the camera is the central component of the app we need a way to navigate through the app that does not cover the camera preview.
		A navigation drawer is used to create a menu that can be displayed by swiping in from the left side of the screen.
		The navigation drawer currently allows users to navigate back to the login activity, to the results activity, or select an image from their phones gallery.

		The login and results activity are mostly skeleton activities.
		Clicking the login button on the login activity just sends you to the main activity without checking the username or password fields.
		The results activity just shows an example image from Ethan's work.

	\subsection{What Is Left}
		The largest piece of remaining functionality is a connection with our server.
		The app will need to be able to send images and data for login purposes.
		Results also need to be sent from the server to be displayed in the app.
		Once a connection is established an account creation activity needs to be created as well.

		When all the functionality is working the UI and general look of the app needs to be made to look like a professional app.
		The navigation drawer needs a lot of adjustments to its size and each element within it needs to be stylized.
		Buttons need to be changed to match a more consistent style.
		A logo to display and more use of icons would go a long way in making the app look cleaner.

		The results activity needs a lot of work once the exact output from the seed identification algorithm is ironed out.
		A list is needed to show all the current user's sample results.
		A single result will consist of an image marked by the algorithm and text describing what the algorithm found.

	\subsection{Issues}
		The most difficult part so far was just getting started.
		I first wanted to get a preview from the phone camera but the app kept immediately crashing from what seemed like a completely unrelated issue.
		The app could not find the startup activity despite it clearly existing where it should be and the IDE being able to find it.
		I read through a lot of Stackoverflow posts but none of them helped.
		Finally, I created a brand new project and copied the camera code to it and it worked just fine.
		The initial project had to be scrapped and created anew; thankfully I caught this early and not much except for time was lost.
		
		

\bibliographystyle{ieeetr}
\bibliography{bib}


\end{document}