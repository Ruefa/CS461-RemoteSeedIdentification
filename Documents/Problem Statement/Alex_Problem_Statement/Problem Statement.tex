\documentclass[onecolumn, draftclsnofoot,10pt, compsoc]{IEEEtran}
\usepackage{graphicx}
\usepackage{url}
\usepackage{setspace}

\usepackage{geometry}
\geometry{textheight=9.5in, textwidth=7in}

% 1. Fill in these details
\def \CapstoneTeamName{		The Cleverly Named Team}
\def \CapstoneTeamNumber{		12}
\def \GroupMemberOne{			Quanah Green}
\def \GroupMemberTwo{			Alex Ruef}
\def \GroupMemberThree{			Ethan Takla}
\def \CapstoneProjectName{		Remote Seed Identification}
\def \CapstoneSponsorCompany{	Crop and Soil Science Department, OSU}
\def \CapstoneSponsorPerson{		Dan Curry}

% 2. Uncomment the appropriate line below so that the document type works
\def \DocType{		Problem Statement
				%Requirements Document
				%Technology Review
				%Design Document
				%Progress Report
				}
			
\newcommand{\NameSigPair}[1]{\par
\makebox[2.75in][r]{#1} \hfil 	\makebox[3.25in]{\makebox[2.25in]{\hrulefill} \hfill		\makebox[.75in]{\hrulefill}}
\par\vspace{-12pt} \textit{\tiny\noindent
\makebox[2.75in]{} \hfil		\makebox[3.25in]{\makebox[2.25in][r]{Signature} \hfill	\makebox[.75in][r]{Date}}}}
% 3. If the document is not to be signed, uncomment the RENEWcommand below
\renewcommand{\NameSigPair}[1]{#1}

%%%%%%%%%%%%%%%%%%%%%%%%%%%%%%%%%%%%%%%
\begin{document}
\begin{titlepage}
    \pagenumbering{gobble}
    \begin{singlespace}
    	%\includegraphics[height=4cm]{coe_v_spot1}
        \hfill 
        % 4. If you have a logo, use this includegraphics command to put it on the coversheet.
        %\includegraphics[height=4cm]{CompanyLogo}   
        \par\vspace{.2in}
        \centering
        \scshape{
            \huge CS Capstone \DocType \par
            {\large\today}\par
            \vspace{.5in}
            \textbf{\Huge\CapstoneProjectName}\par
            \vfill
            {\large Prepared for}\par
            \Huge \CapstoneSponsorCompany\par
            \vspace{5pt}
            {\Large\NameSigPair{\CapstoneSponsorPerson}\par}
            {\large Prepared by }\par
            Group\CapstoneTeamNumber\par
            % 5. comment out the line below this one if you do not wish to name your team
            %\CapstoneTeamName\par 
            \vspace{5pt}
            {\Large
                \NameSigPair{\GroupMemberOne}\par
                \NameSigPair{\GroupMemberTwo}\par
                \NameSigPair{\GroupMemberThree}\par
            }
            \vspace{20pt}
        }
        \begin{abstract}
        % 6. Fill in your abstract    
        %This document is written using one sentence per line.
        %This allows you to have sensible diffs when you use \LaTeX with version control, as well as giving a quick visual test to see if sentences are too short/long.
        %If you have questions, ``The Not So Short Guide to LaTeX'' is a great resource (\url{https://tobi.oetiker.ch/lshort/lshort.pdf})
	An inefficiency has been discovered in the seed identification process.
	Seed cleaning facilities must send cleaned seeds to a seed lab for them to be analyzed.
	The cleaning facilities must then wait for the results to come back.
	The seeds are analyzed by hand at the seed lab further adding to the time delay.
	Our solution to this problem will reduce or eliminate both the time required to send the sample and the time required to analyze the sample.
	We will use a machine learning algorithm to teach a processor how to identify different seed types in an image containing thousands of seeds.
	This will greatly speed up the analysis time.
	These images will be sent to the process from Ipads out in the field.
	By sending images over network we can effectively eliminate the time required to transport the seed.
        \end{abstract}     
    \end{singlespace}
\end{titlepage}
\newpage
\pagenumbering{arabic}
\tableofcontents
% 7. uncomment this (if applicable). Consider adding a page break.
%\listoffigures
%\listoftables
\clearpage

% 8. now you write!
\section{Description}
This project aims to resolve a bottleneck in the seed identification process.
In the current system seeds need to be cleaned and then sent to a seed lab for analysis.
The cleaning machines need to be shut down until the results come back from the lab.
The seeds are physically sent to a lab to be analyzed by hand.
Physically sending the seeds to a lab and analyzing them takes a significant amount of time.
The cleaning process has to stall and wait for the results creating a time deficiency.
The results contain information on how well the seeds were cleaned and how many unwanted seeds were in the sample.
How well a seed is cleaned is defined by how much dirt and other matter is left on the seed.
The information is then sent back to the cleaning facility where it is used to improve the cleaning process.

\section{Solution}
To solve this issue, we will eliminate the need to physically send the seeds to a lab.
Users will take pictures of seeds using an Ipad or high quality camera and send them to a server.
A camera rig may be required to help users take better images of the seeds.
These images will ideally contain 2500 seeds each.
If the image quality is not good enough to contain 2500 seeds then less seeds can be included in an image but more images will have to be sent.
The number of seeds in each imagine and whether we need a camera rig or not depends on the base resolution of the camera and how powerful our identification algorithm is.
The server will use machine learning to identify each seed and some information about them.
A Jetson TX2 processor will be used to run the algorithm.
This information will contain things like how well each seed was cleaned and how many unwanted seeds there are.
The information is then stored in a database to aid our server with future identifications.
A PDF is generated based on the information from the sample and is sent back to the cleaning facility so they modify their equipment and continue cleaning.

\section{Performance Metrics}
The base requirements for remote seed identification are as follows.
Users in the field must be able to take images using an Ipad and send them remotely to a server.
They do not necessarily have to be Ipads but that is what we are using as of now.
These Images will ideally contain 2500 seeds.
If the cameras do not have enough resolution for 2500 seeds then less seeds can be sent in a single image but more images will need to be sent for a total of 2500 seeds.
The server will contain our algorithm for identifying seeds.
The algorithm will ideally be able to handle 2500 seeds in a single image but as I have said before less seeds can be sent in a single image.
The algorithm determines what seeds are present, how clean they are, and how many unwanted seeds are present.
The results of the algorithm are stored in a database and emailed back to the users who took the images.


\end{document}