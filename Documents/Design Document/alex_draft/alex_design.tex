\documentclass[onecolumn, draftclsnofoot,10pt, compsoc]{IEEEtran}
\usepackage{graphicx}
\usepackage{url}
\usepackage{setspace}
\usepackage{cite}

\usepackage{geometry}
\geometry{textheight=9.5in, textwidth=7in}

% 1. Fill in these details
\def \CapstoneTeamName{		Remote Seed}
\def \CapstoneTeamNumber{		12}
\def \GroupMemberOne{			Quanah Green}
\def \GroupMemberTwo{			Alex Ruef}
\def \GroupMemberThree{			Ethan Takla}
\def \CapstoneProjectName{		Remote Seed Identification}
\def \CapstoneSponsorCompany{	Crop and Soil Science Department, OSU}
\def \CapstoneSponsorPerson{		Dan Curry}

% 2. Uncomment the appropriate line below so that the document type works
\def \DocType{		%Problem Statement
				%Requirements Document
				%Technology Review
				Design Document
				%Progress Report
				}
			
\newcommand{\NameSigPair}[1]{\par
\makebox[2.75in][r]{#1} \hfil 	\makebox[3.25in]{\makebox[2.25in]{\hrulefill} \hfill		\makebox[.75in]{\hrulefill}}
\par\vspace{-12pt} \textit{\tiny\noindent
\makebox[2.75in]{} \hfil		\makebox[3.25in]{\makebox[2.25in][r]{Signature} \hfill	\makebox[.75in][r]{Date}}}}
% 3. If the document is not to be signed, uncomment the RENEWcommand below
\renewcommand{\NameSigPair}[1]{#1}

%%%%%%%%%%%%%%%%%%%%%%%%%%%%%%%%%%%%%%%
\begin{document}
\begin{titlepage}
    \pagenumbering{gobble}
    \begin{singlespace}
    	%\includegraphics[height=4cm]{coe_v_spot1}
        \hfill 
        % 4. If you have a logo, use this includegraphics command to put it on the coversheet.
        %\includegraphics[height=4cm]{CompanyLogo}   
        \par\vspace{.2in}
        \centering
        \scshape{
            \huge CS Capstone \DocType \par
            {\large\today}\par
            \vspace{.5in}
            \textbf{\Huge\CapstoneProjectName}\par
            \vfill
            {\large Prepared for}\par
            \Huge \CapstoneSponsorCompany\par
            \vspace{5pt}
            {\Large\NameSigPair{\CapstoneSponsorPerson}\par}
            {\large Prepared by }\par
            Group\CapstoneTeamNumber\par
            % 5. comment out the line below this one if you do not wish to name your team
            \CapstoneTeamName\par 
            \vspace{5pt}
            {\Large
                \NameSigPair{\GroupMemberOne}\par
                \NameSigPair{\GroupMemberTwo}\par
                \NameSigPair{\GroupMemberThree}\par
            }
            \vspace{20pt}
        }
        \begin{abstract}
        		

        \end{abstract}     
    \end{singlespace}
\end{titlepage}
\newpage
\pagenumbering{arabic}
\tableofcontents
% 7. uncomment this (if applicable). Consider adding a page break.
%\listoffigures
%\listoftables
\clearpage
\section{Mobile App}
	The mobile app for our project will need to perform three main features.
	The app will need to take pictures or access the users images, send the images to our Jetson processor, and receive back data from the Jetson.
	We will be using the Xamarin UI framework to build the app.
	The code will be written in C\# since Xamarin works best with C\#.

	\subsection{First Time Running}
		When the app is ran for the first time the user will have to login or create a new account.
		These accounts will be used to track who is sending seed samples so the results can be sent back to the right person.
		The fields required to create a account will be name, email, and password??
		Communication with the server will be done using the Volley library to send HTTP requests to an API on the server.
		Once users have successfully logged in they will be shown a home or a navigation page.

	\subsection{Navigation}
		If a user has logged in previously the navigation page will be the first page they see.
		This page will be used to send users to either the send seed sample section or the section where they can see results.
		There will also be an options button should we need it.

	\subsection{Send Sample}
		There will likely be multiple images for a single sample.
		Users will first have input how many images will be sent in the sample.
		Once that is done the user needs to present an image to be sent.
		Selecting an image can be done from the users images or by utilizing the phones camera in the app.
		The images will be sent by HTTP requests to our server which will then be processed by the Jetson.

	\subsection{Results}
		Users can be sent a notification and/or an email when the results are done processing.
		If processing is quick enough users will be send to the results page after uploading a sample.
		
		

\bibliographystyle{ieeetr}
\bibliography{bib}


\end{document}