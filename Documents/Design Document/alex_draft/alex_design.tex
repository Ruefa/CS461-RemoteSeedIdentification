\documentclass[onecolumn, draftclsnofoot,10pt, compsoc]{IEEEtran}
\usepackage{graphicx}
\usepackage{url}
\usepackage{setspace}
\usepackage{cite}

\usepackage{geometry}
\geometry{textheight=9.5in, textwidth=7in}

% 1. Fill in these details
\def \CapstoneTeamName{		Remote Seed}
\def \CapstoneTeamNumber{		12}
\def \GroupMemberOne{			Quanah Green}
\def \GroupMemberTwo{			Alex Ruef}
\def \GroupMemberThree{			Ethan Takla}
\def \CapstoneProjectName{		Remote Seed Identification}
\def \CapstoneSponsorCompany{	Crop and Soil Science Department, OSU}
\def \CapstoneSponsorPerson{		Dan Curry}

% 2. Uncomment the appropriate line below so that the document type works
\def \DocType{		%Problem Statement
				%Requirements Document
				%Technology Review
				Design Document
				%Progress Report
				}
			
\newcommand{\NameSigPair}[1]{\par
\makebox[2.75in][r]{#1} \hfil 	\makebox[3.25in]{\makebox[2.25in]{\hrulefill} \hfill		\makebox[.75in]{\hrulefill}}
\par\vspace{-12pt} \textit{\tiny\noindent
\makebox[2.75in]{} \hfil		\makebox[3.25in]{\makebox[2.25in][r]{Signature} \hfill	\makebox[.75in][r]{Date}}}}
% 3. If the document is not to be signed, uncomment the RENEWcommand below
\renewcommand{\NameSigPair}[1]{#1}

%%%%%%%%%%%%%%%%%%%%%%%%%%%%%%%%%%%%%%%
\begin{document}
\begin{titlepage}
    \pagenumbering{gobble}
    \begin{singlespace}
    	%\includegraphics[height=4cm]{coe_v_spot1}
        \hfill 
        % 4. If you have a logo, use this includegraphics command to put it on the coversheet.
        %\includegraphics[height=4cm]{CompanyLogo}   
        \par\vspace{.2in}
        \centering
        \scshape{
            \huge CS Capstone \DocType \par
            {\large\today}\par
            \vspace{.5in}
            \textbf{\Huge\CapstoneProjectName}\par
            \vfill
            {\large Prepared for}\par
            \Huge \CapstoneSponsorCompany\par
            \vspace{5pt}
            {\Large\NameSigPair{\CapstoneSponsorPerson}\par}
            {\large Prepared by }\par
            Group\CapstoneTeamNumber\par
            % 5. comment out the line below this one if you do not wish to name your team
            \CapstoneTeamName\par 
            \vspace{5pt}
            {\Large
                \NameSigPair{\GroupMemberOne}\par
                \NameSigPair{\GroupMemberTwo}\par
                \NameSigPair{\GroupMemberThree}\par
            }
            \vspace{20pt}
        }
        \begin{abstract}
        		

        \end{abstract}     
    \end{singlespace}
\end{titlepage}
\newpage
\pagenumbering{arabic}
\tableofcontents
% 7. uncomment this (if applicable). Consider adding a page break.
%\listoffigures
%\listoftables
\clearpage
\section{Mobile App}
	The mobile app for our project will need to perform three main features.
	The app will need to take pictures or access the users images, send the images to our Jetson processor, and receive back data from the Jetson.
	We will be using the Xamarin UI framework to build the app.
	The code will be written in C\# since Xamarin works best with C\# and C\# is commonly used for app development.
	The app will be developed for android devices with an iOs app as a stretch goal.
	Using a UI framework will make porting the android app to iOs much easier.

	\subsection{First Time Running}
		When the app is ran for the first time the user will have to login or create a new account.
		These accounts will be used to track who is sending seed samples so the results can be sent back to the right person.
		Users will use an email and password to login.
		The fields required to create a account will be name, email, and password.
		Communication with the server will be done using the Volley library to send HTTP requests to an API on the server.
		Once users have successfully logged in they will be shown a home or a navigation page.

	\subsection{Navigation}
		If a user has logged in previously the navigation page will be the first page they see.
		This page will be used to send users to either the send seed sample section or the section where they can see results.
		There will also be a button to a help page to tell users how to use the app and a little about what it does.
		There will also be an options button should we need it.

	\subsection{Send Seed Sample}
		There will likely be multiple images for a single sample.
		Users will first have input how many images will be sent in the sample.
		Once that is done the user needs to present an image to be sent.
		Selecting an image can be done from the users images or directly from the phones camera.
		The phones system can handle this for us we just need to make the right API calls.
		The images will be sent by HTTP requests to our server which will then be processed by the Jetson.
		After the images are sent the user will see a message informing them if the image upload was successful or not.
		The user can then either be sent to the results page or told to wait for the results and check their email.

	\subsection{Results}
		Users can be sent a notification and/or an email when the results are done processing.
		If processing is quick enough users will be send to the results page after uploading a sample.
		Users will be able to see a list of previous samples sorted by date with the most recent at the forefront.
		The user must select the sample they want before the results are displayed.
		Sample output display will depend on what is contained in the results.
		Users should also receive a PDF of the results in an email.
		

\bibliographystyle{ieeetr}
\bibliography{bib}


\end{document}