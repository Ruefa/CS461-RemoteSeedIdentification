\documentclass[onecolumn, draftclsnofoot,10pt, compsoc]{IEEEtran}
\usepackage{graphicx}
\usepackage{url}
\usepackage{setspace}
\usepackage{cite}

\usepackage{geometry}
\geometry{textheight=9.5in, textwidth=7in}

% 1. Fill in these details
\def \CapstoneTeamName{		Remote Seed}
\def \CapstoneTeamNumber{		12}
\def \GroupMemberOne{			Quanah Green}
\def \GroupMemberTwo{			Alex Reuf}
\def \GroupMemberThree{			Ethan Takla}
\def \CapstoneProjectName{		Remote Seed Identification}
\def \CapstoneSponsorCompany{	Crop and Soil Science Department, OSU}
\def \CapstoneSponsorPerson{		Dan Curry}

% 2. Uncomment the appropriate line below so that the document type works
\def \DocType{		%Problem Statement
				%Requirements Document
				%Technology Review
				Design Document
				%Progress Report
				}
		
\newcommand{\NameSigPair}[1]{\par
\makebox[2.75in][r]{#1} \hfil 	\makebox[3.25in]{\makebox[2.25in]{\hrulefill} \hfill		\makebox[.75in]{\hrulefill}}
\par\vspace{-12pt} \textit{\tiny\noindent
\makebox[2.75in]{} \hfil		\makebox[3.25in]{\makebox[2.25in][r]{Signature} \hfill	\makebox[.75in][r]{Date}}}}
% 3. If the document is not to be signed, uncomment the RENEWcommand below
\renewcommand{\NameSigPair}[1]{#1}

%%%%%%%%%%%%%%%%%%%%%%%%%%%%%%%%%%%%%%%
\begin{document}
\begin{titlepage}
    \pagenumbering{gobble}
    \begin{singlespace}
    	%\includegraphics[height=4cm]{coe_v_spot1}
        \hfill 
        
        % 4. If you have a logo, use this includegraphics command to put it on the coversheet.
        %\includegraphics[height=4cm]{CompanyLogo}   
        \par\vspace{.2in}
        \centering
        \scshape{
            \huge CS Capstone \DocType \par
            {\large\today}\par
            \vspace{.5in}
            \textbf{\Huge\CapstoneProjectName}\par
            \vfill
            {\large Prepared for}\par
            \Huge \CapstoneSponsorCompany\par
            \vspace{5pt}
            {\Large\NameSigPair{\CapstoneSponsorPerson}\par}
            {\large Prepared by }\par
            Group\CapstoneTeamNumber\par
            % 5. comment out the line below this one if you do not wish to name your team
            \CapstoneTeamName\par 
            \vspace{5pt}
            {\Large
                \NameSigPair{\GroupMemberOne}\par
                \NameSigPair{\GroupMemberTwo}\par
                \NameSigPair{\GroupMemberThree}\par
            }
            \vspace{20pt}
        }
        \begin{abstract}
        % 6. Fill in your abstract    
        \end{abstract}     
    \end{singlespace}
\end{titlepage}
\newpage
\pagenumbering{arabic}
\tableofcontents
% 7. uncomment this (if applicable). Consider adding a page break.
%\listoffigures
%\listoftables
\clearpage

% 8. now you write!
\section{Overview}
    \subsection{Scope}
    \subsection{Purpose}
    \subsection{Intended Audience}
    \subsection{Conformance}

\section{Definitions}

\section{Design}
    \subsection{Server}
        A server will be needed to facilitate transfer of images and reports between the mobile app and the Jetson processor where the actual analysis will happen.
    In order to reduce costs and simplify the design, the server will be run on the Jetson itself.
    A dedicated server could be implemented at a later time, but until the core functionality of the seed identification process has been shown effective there is no reason to spend time and money on a dedicated server.

    The server will be implemented in Python 3 using the http.server package.
    As we will only be serving simple requests, the built in python http.server package will be sufficient to create a functional server fairly easily.
    When the user logs in through the mobile app a connection will be opened to the server, which will validate their log in information in the database, and upon successful login update their most recent IP address in the database.
    When the user decides to upload images to create a report, a connection to the Server will again be opened.
    The images will be sent to the Jetson over the open connection, and passed directly in to the seed identification API.
    Once the analysis is finished, a connection will be opened back to the client at the IP address stored in the database, and a report will be sent back to them.

    \subsection{Database}
    A database will need to be maintained that includes login information and addresses for sending back reports.
    The database will need be accessed for validating user login information.
    Furthermore, each time a logged in client establishes a connection to the server, the database will be updated with their most recent IP address.
    Once a job is processed and a report is generated, the address of the client who requested the analysis will be looked up in the database, and the report will be sent to that address.

    The database will be implemented in the Jetson application using SQLite.
    This will be accessed in the Server program using the PonyORM python package.
    This will allow for easy database access from the python based server.


    \subsection{Report Generation}
    When an analysis is run, the results need to be recorded and conveyed in an easily readable format.
    Once the analysis is completed on the Jetson, the results will be passed as numerical arguments to a function in the server interface, which will use the open source ReportLab python library to generate a PDF report, to be sent to the client.
    






\end{document}
