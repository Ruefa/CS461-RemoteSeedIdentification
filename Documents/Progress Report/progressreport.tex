\documentclass[onecolumn, draftclsnofoot,10pt, compsoc]{IEEEtran}
\usepackage{graphicx}
\usepackage{url}
\usepackage{setspace}
\usepackage{biblatex}
\addbibresource{bibliography.bib}



\usepackage{geometry}
\geometry{textheight=9.5in, textwidth=7in}

% 1. Fill in these details
\def \CapstoneTeamName{		The Cleverly Named Team}
\def \CapstoneTeamNumber{		12}
\def \GroupMemberOne{			Ethan Takla}
\def \CapstoneProjectName{		Remote Seed Identification}
\def \CapstoneSponsorCompany{	Crop and Soil Science Department, OSU}
\def \CapstoneSponsorPerson{		Dan Curry}

% 2. Uncomment the appropriate line below so that the document type works
\def \DocType{		%Problem Statement
				Technology Review
				%Technology Review
				%Design Document
				%Progress Report
				}
			
\newcommand{\NameSigPair}[1]{\par
\makebox[2.75in][r]{#1} \hfil 	\makebox[3.25in]{\makebox[2.25in]{\hrulefill} \hfill		\makebox[.75in]{\hrulefill}}
\par\vspace{-12pt} \textit{\tiny\noindent
\makebox[2.75in]{} \hfil		\makebox[3.25in]{\makebox[2.25in][r]{Signature} \hfill	\makebox[.75in][r]{Date}}}}
% 3. If the document is not to be signed, uncomment the RENEWcommand below
\renewcommand{\NameSigPair}[1]{#1}

%%%%%%%%%%%%%%%%%%%%%%%%%%%%%%%%%%%%%%%
\begin{document}
\begin{titlepage}
    \pagenumbering{gobble}
    \begin{singlespace}
    	%\includegraphics[height=4cm]{coe_v_spot1}
        \hfill 
        % 4. If you have a logo, use this includegraphics command to put it on the coversheet.
        %\includegraphics[height=4cm]{CompanyLogo}   
        \par\vspace{.2in}
        \centering
        \scshape{
            \huge CS Capstone \DocType \par
            {\large\today}\par
            \vspace{.5in}
            \textbf{\Huge\CapstoneProjectName}\par
            \vfill
            {\large Prepared for}\par
            \Huge \CapstoneSponsorCompany\par
            \vspace{5pt}
            {\Large\NameSigPair{\CapstoneSponsorPerson}\par}
            {\large Prepared by }\par
            Group\CapstoneTeamNumber\par
            % 5. comment out the line below this one if you do not wish to name your team
            %\CapstoneTeamName\par 
            \vspace{5pt}
            {\Large
                \NameSigPair{\GroupMemberOne}\par
            }
            \vspace{20pt}
        }
    \end{singlespace}
\end{titlepage}
\newpage
\pagenumbering{arabic}
\tableofcontents
% 7. uncomment this (if applicable). Consider adding a page break.
%\listoffigures
%\listoftables
\clearpage

\section{Purposes and goals}

Currently, the seed identification process is tedious and inefficient. Seed analysts can look at up to 25,000 seeds a day using nothing more than a conveyor belt and a microscope, which is quite an archaic process in a world where machine learning and artificial intelligence are beginning to make their way into every aspect of technology. In addition to analyzing seeds in the lab, researchers often find themselves needing to take samples in the field. Currently, the best way to do this is to send the samples to a 3rd party company, which takes excess time and money. With these glaring issues in mind, it becomes obvious that a solution for quick and efficient seed identification is desperately needed at Oregon State. The purpose of the remote seed project is to do just this, and additionally potentially disrupt the seed industry as a whole. Due to the fact that this project is more research oriented, a set of minimum basic goals is first defined, and then subsequently a set of ultimate goals that we would like to reach. At a basic minimum, the seed identify should be able to provide a yes/no output on each seed in a 500 seed sample. This essentially means that it will be able to tell if certain seeds aren't of a specific type, and will allow a metric on the purity of the sample. As a stretch goal, the identifier will be able to discern different seed species, which will provide even more insight for the seed analysts. 

For the mobile application, our goals are to provide an intuitive way for analysts to take seed samples, as well as saving and retrieving them for later analysis. Every individual that uses the service will have a user account which they use to access their previous reports, and additionally generate reports in the forms of PDFs that can be sent through email.

\section{State of the Project}

This term, the majority of time has been spent figuring out the implementation details of the project, in addition to getting seed training data to train out classifiers. Currently, we have been able to implement basic image classifiers using pre-trained networks to make ourselves more familiar with the process of training and running the neural networks. We have also configured OpenCV and Caffe to run using NVIDIA GPU's, meaning that when we go to train with real data the process will be much faster. Due to the fact that we are unsure if a phone camera will be able to provide detailed enough photos for the classifier, we have begun research methods to obtain higher-resolution photos from multiple lower-resolution images. These methods include Super-resolution and image stitching algorithms, which have proven difficult to use due to their high amounts of resource usage. Currently, we have obtained nearly 6000 images of both \textit{Tall Fescue} and \textit{Perennial Ryegrass} seeds. Over winter break, the plan is to experiment with these images and determine how accurate the classifier can become. 

\section{Current issues}

Although no glaring issue have been encountered so far, there are some concerns regarding the resolution of the seed image training data. The images we obtained were limited by the camera we used, which means that they might not be high quality enough to provide consistent results. In the case that new image data sets are needed, Carrie Lewis of the OSU seed lab has offered to let us use her equiptment. 

\clearpage

\section{Weekly Summaries}
	A week-by-week summary of our progress, goals, and problems is provided below. 
	\subsection{Week 1}
	
	    \subsubsection{Plans}
        We need to finalize our project preferences and make sure they're submitted properly. In addition to this, OneNote must be properly formatted along with the professional biography. 
        
    	\subsubsection{Progress}
    	We didn't do much this week other than learn about the class and the process that comes along with it. Many companies use OneNote as their primary note-taking software, so it is important to get familiar with it. All of our plans were completed.

    \subsection{Week 2}
        \subsubsection{Plans}
        Meet with our client, start the individual draft of the problem statement and
        determine the primary way of communication between group members
        \subsubsection{Progress}
        This week wasn't too work intensive, and was rather for getting to know the teammates and becoming familiar with our project assignment. Additionally, the meeting with Dan got pushed back to the 9th.
        \subsubsection{Problems}
        We at first had a hard time finding time between schedules, but otherwise there weren't any glaring issues this week.
    \subsection{Week 3}
        \subsubsection{Plans}
        Meeting with Dan planned for Monday at 11 AM, compile individual problem statement rough draft, Google hangout with our TA, and make a GitHub

        \subsubsection{Progress}
        Things really started moving this week, as we were able to meet with Dan, get a better understanding of the problem, and write our problem statements accordingly. Alex created a Github for us and we pushed each of our individual statements to GitHub.
    \subsection{Week 4}
        \subsubsection{Plans}
        Meet with group to discuss individual problem statements and combine them, turn in finalized problem statement, and have Dan send the professors an email verification that he got it. Additionally, everyone needs a picture of themselves uploaded to OneNote
        \subsubsection{Progress}
        This was another productive week, as we created a polished problem statement and really got a feel for what problems we're trying to solve. We spent a solid amount of time doing some R\&D for the AI in the back-end and made some good discoveries, such as the fact OpenCV's Super Resolution algorithm consumes too many resources when starting with high-res photos.

    \subsection{Week 5}
        \subsubsection{Plans}
        Make a rough draft of the requirements document, and meet with team and brainstorm functions and requirements for the document
        \subsubsection{Progress}
        This was a relatively slow week. We got a rough draft of the problem statement done and had a Google hangout to brainstorm ideas.

    \subsection{Week 6}
        \subsubsection{Plans}
        Work diligently to finish the requirements document and have it sent in by 11:59 PM on Friday and send Dan a rough draft of the requirements document for feedback and incorporate his feedback.
        
        \subsubsection{Progress}
        No significant problems this week, but rather some good feedback from Dan about what he wants in the product. Another issue that was raised was that we might need add storage to the Jetson TX2 setup so that clients can store important seed analysis reports.

    \subsection{Week 7}
        \subsubsection{Plans}
        Start figuring out what camera we're going to take seed pictures with (possibly Dan's camera), and begin working on the technology review rough draft
        \subsubsection{Progress}
        This week, we were able to set up a meeting with Dan to get the Camera, as well as beginning on our technology review

    \subsection{Week 8}
    
        \subsubsection{Plans}
        Meet with Dan to get the Camera for taking pictures, start getting good images of the seeds, get the first draft of the tech. review in, and begin working on the final draft of the tech review.
        \subsubsection{Progress}
        Besides the technology review rough draft, this was a slow week that had some issues involving the camera. 
        \subsubsection{Problems}
        The camera requires a dedicated PCIE card that is installed on an OSU computer, so we won't be able to take the camera home for picture taking.
    \subsection{Week 9}
    
        \subsubsection{Plans}
        Finish the final draft of the tech review, and set up a time to go in and take pictures of seeds with Dan.
        
        \subsubsection{Progress}
        We Finished the final draft of the technology review. Additionally, we built a computer and installed Mint on it so we won't have to use a Virtual Box anymore to test the machine learning. 

    \subsection{Week 10}
        \subsubsection{Plans}
        Turn in the final design document, and go into Dan's office to take seed pictures.
        
        \subsubsection{Progress}
        We were able to get the design document done, and although we weren't able to take seed photos, we found nearly 6000 images from the previous team. In the case that the photos we found were too low quality, we set up a contact with Carrie from the OSU Seed Lab to take better ones. 
        
        \subsubsection{Problems}
        Dan's camera produced lower-quality images than we thought, so we weren't able to get the pictures


\clearpage

\section{Retrospective}

\begin{table}[ht]
	\begin{tabular}{|p{0.3\linewidth}|p{0.3\linewidth}|p{0.3\linewidth}|}
    \hline
    Positives & Deltas & Actions \\ \hline
    So far we have learned a lot about machine learning and the tools required to build an Andorid application &	
    All of our group members have been very busy, and it has been hard to dedicate a lot of time to this project. &
    Over winter break, we are going to start implementing the back-end. During winter term, we are going to meet at least once a week for 2 hours to get things done. \\ \hline
    We have become close as a group, and know each others strengths and weaknesses now. This allows us to work more efficiently. & We're excited to work together to begin putting all of the pieces of the project together. This winter term will be a lot of work when it comes to implementation. & We need to further improve our time line by making it more specific.  \\ \hline
    The team and Daniel have gotten along very well, and he is excited to be working with us. 
    && \\ \hline
    
    \end{tabular}
\end{table}

\printbibliography

\end{document}